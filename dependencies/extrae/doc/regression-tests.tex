\chapter{Regression tests}\label{cha:RegressionTests}

\TRACE includes a battery of regression tests to evaluate whether recent versions of the instrumentation package keep their compatibility and that new changes on it have not introduced new faults.
These tests are meant to be executed in the same machine that compiled \TRACE and they are not intended to support its execution through batch-queuing systems nor cross-compilation processes.
To invoke the tests, simply run from the terminal the following command:

\graybox{\texttt{make check}}

after the configuration and building process. It will automatically invoke all the tests one after another and will produce several summaries.

These tests are divided into different categories that stress different parts of \TRACE.
The current categories tested include, but are not limited to:
\begin{itemize}
\item Clock routines
\item Instrumentation support
\begin{itemize}
  \item Event definition in the PCF from the \TRACE API
  \item pthread instrumentation
  \item MPI instrumentation
  \item Java instrumentation
\end{itemize}
\item Merging process (i.e. \texttt{mpi2prv})
\item Callstack unwinding (either using libunwind library or backtrace)
\item Performance hardware counters through PAPI library
\item XML parsing through libxml2
\end{itemize}

These tests will change during the development of \TRACE.
If the reader has a particular suggestion on a particular test, please consider to send it to \texttt{tools@bsc.es} for its consideration.

